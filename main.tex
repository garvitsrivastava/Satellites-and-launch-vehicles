\documentclass{article}
\usepackage[utf8]{inputenc}

\title{Satellites and launch vehicles}
\author{Garvit Srivastava}

\begin{document}

\maketitle
\begin{abstract}
    Satellites and launch vehicles both fall into the same category     "Spacecraft". A spacecraft is a vehicle which is used for travelling in space, and space is not necessary always the outer space, 
\end{abstract}
\section{Introduction}
\subsection{Launch Vehicles}
A launch vehicle or carrier rocket is a rocket used to carry a payload from Earth's surface through outer space, either to another surface point (Sub-orbital transportation), or into space (Earth orbit or beyond). A launch system includes the launch vehicle, launch pad, vehicle assembly and fuelling systems, range safety, and other related infrastructure

Suborbital launch vehicles include ballistic missiles, sounding rockets, and various crewed systems designed for space tourism or high-speed transport. Orbital or escape launch vehicles must be much more powerful and typically incorporate two to four rocket stages to provide sufficient delta-v (change in velocity) performance. Various rocket fuels are used, including solid rocket boosters and cryogenic fuels fed to rocket engines.
\subsection{Satellites}
\end{document}
